%% start of file `template.tex'.
%% Copyright 2006-2010 Xavier Danaux (xdanaux@gmail.com).
%
% This work may be distributed and/or modified under the
% conditions of the LaTeX Project Public License version 1.3c,
% available at http://www.latex-project.org/lppl/.


\documentclass[11pt,a4paper]{../template/moderncv}

% moderncv themes
\moderncvtheme[blue]{classic}                 % optional argument are 'blue' (default), 'orange', 'red', 'green', 'grey' and 'roman' (for roman fonts, instead of sans serif fonts)
%\moderncvtheme[green]{classic}                % idem

% character encoding
\usepackage[utf8]{inputenc}                   % replace by the encoding you are using

% adjust the page margins
\usepackage[scale=0.8]{geometry}
%\setlength{\hintscolumnwidth}{3cm}						% if you want to change the width of the column with the dates
%\AtBeginDocument{\setlength{\maketitlenamewidth}{6cm}}  % only for the classic theme, if you want to change the width of your name placeholder (to leave more space for your address details
%\AtBeginDocument{\recomputelengths}                     % required when changes are made to page layout lengths

% personal data
\firstname{Walter}
\familyname{Vargas}
\title{Walter Vargas Curriculum vitae et studiorum}               % optional, remove the line if not wanted
\address{Manzana 18, Parcela 24 Calle 7 los Sauces Altos de Paramillo}{San Cristóbal - Edo Táchira}    % optional, remove the line if not wanted
\mobile{+58(416)5023755}                    % optional, remove the line if not wanted
%\phone{}                      % optional, remove the line if not wanted
\fax{+58(276)3571649}                          % optional, remove the line if not wanted
\email{walter@covetel.com.ve \\ waltervargas@gmail.com \\ wvargas@cpan.org}                      % optional, remove the line if not wanted
\homepage{goo.gl/eTlJu}                % optional, remove the line if not wanted
\extrainfo{} % optional, remove the line if not wanted
\photo[64pt]{picture}                         % '64pt' is the height the picture must be resized to and 'picture' is the name of the picture file; optional, remove the line if not wanted
\quote{El mayor enemigo del conocimiento no es la ignorancia, sino la ilusión
del conocimiento -- Stephen Hawking}                 % optional, remove the line if not wanted

% to show numerical labels in the bibliography; only useful if you make citations in your resume
\makeatletter
\renewcommand*{\bibliographyitemlabel}{\@biblabel{\arabic{enumiv}}}
\makeatother

% bibliography with mutiple entries
%\usepackage{multibib}
%\newcites{book,misc}{{Books},{Others}}

%\nopagenumbers{}                             % uncomment to suppress automatic page numbering for CVs longer than one page
%----------------------------------------------------------------------------------
%            content
%----------------------------------------------------------------------------------
\begin{document}
\maketitle

\section{Información Personal}
\cvitem{Lugar de Nacimiento}{\small Caracas - República Bolivariana de Venezuela }
\cvitem{Fecha de Nacimiento}{\small 17 de Febrero de 1985 }
\cvitem{Estado Civil}{\small Soltero}
\cvitem{Cédula de Identidad}{\small V-16.612.574}
\cvitem{GNU/Linux User ID}{\small 268566}
\cvitem{Skype}{\small waltervargasm}

\section{Educación}
%\cventry{year--year}{Degree}{Institution}{City}{\textit{Grade}}{Description}  % arguments 3 to 6 can be left empty
\cventry{Actualidad}{Pregrado}{Instituto Universitario de
Tecnología Agro Industrial Región los Andes}{San Cristóbal}{Ingeniería
Informática}{Esperando obtener el título para finales del 2011}

\cventry{2002--2009}{Pregrado}{Instituto Universitario de Tecnología Agro
Industrial Región los Andes}{San Cristóbal}{Tecnico Superior en Informática}{}

\section{Lenguajes}
\cvlanguage{Español}{Nativo}{Lectura, Escritura, Comprensión}
\cvlanguage{Ingles}{Intermedio}{Lectura, Escritura, Comprensión}

\section{Habilidades en Computación}
\cvcomputer{\bfseries{Sistemas Operativos}}{
	Debian GNU/Linux (Especialmente), 
	RedHat, 
	Mandriva, 
	Unix System V, 
	Windows NT-2000-XP,
	BeOS,
	Linux Router Project,
	HP-UX,
	Cisco OS
}{\bfseries{Administración}}{
	CFEngine,
	DNS, DHCP,
	OpenLDAP, 
    389 Directory Server, 
	Sendmail, 
	Postfix, 
	Dovecot, 
	Samba, 
	OpenVPN, 
	OpenSSL, 
	OpenCA, 
	Git, 
	Subversion, 
	Trac, 
	Apache (Mod SSL, Mod Perl, Mod PHP), 
	Mailman, 
	Request Tracker, 
	Squid, 
	SquidGuard, 
	Iptables, 
	Ipchains, 
	NMAP,
    PFSense, 
	WebGUI
}
\cvcomputer{\bfseries{Programación}}{
	\textsc{Pascal}, 
	C,
	C++, 
	SQL
}{\bfseries{Scripting}}{
	Perl (Especialmente), 
	JavaScript, 
	Bash, 
	PHP, 
 	Ruby
}
\cvcomputer{\bfseries{Librerías}}{
	LWP, LWP::Simple
	Net::LDAP,
	DBI,
	DBIx::Class,
	Catalyst,
	HTML::FormFu,
	Config::General,
	WWW::Mechanize,
	XML::LibXML,
    Template::Toolkit, 
    Devel::NYTProf, 
	Catalyst::Controller::REST,
	Catalyst::Controller::SOAP,
	Catalyst::Model::SOAP,
	Catalyst::View::GD,
	Catalyst::Model::LDAP,
	Image::Resize,
	GD,
	Moose,
	QT,
	GTK+,
	Statistics::Zscore,
	Data::Dumper,
	Spreadsheet::Read,
    IPC::System::Simple, 
    Win32::FileOp, 
    Win32::AdminMisc,
    File::Basename
}{\bfseries{Tipografía y Diseño}}{
	\LaTeX, 
	Scribus, 
	Inkscape, 
	Gimp, 
    Adobe Photoshop Lightroom
}
\cvcomputer{\bfseries{Desarrollo Web}}{
	Catalyst Framework,
	XML, 
	XLST, 
	XPATH, 
	HTML, 
	XHTML, 
	CSS, 
	AJAX, 
	SOAP,
	WSDL,
	REST,
	Patrones de Diseño
}{\bfseries{Base de Datos}}{
	PostgreSQL, 
	SQLite, 
	Berkeley DB, 
	MySQL, 
	CouchDB
}
\cvcomputer{\bfseries{Virtualizacion}}{
    XEN,
    OpenVZ, 
    KVM, 
    vmware, 
    Proxmox
}{\bfseries{Microsoft}}{
	Windows Server 2000/2003, 
	Windows Management Instrumentation (WMI), 
	Win32::Process::Info, 
    Win32::OLE, 
    Active Directory, 
	Active Directory Service Interfaces (ADSI), 
    Microsoft Exchange, 
}

\section{Habilidades en Electrónica}
\cvcomputer{\bfseries{Instrumentación}}{
	Manejo de Cautin, Multimetro y construcción básica de circuitos impresos.
}{\bfseries{Radio Difusión}}{
	Operación de equipos de radio difusión como consolas mezcladoras, ecualizadores, filtros, y equipos de trasmisión como FM/AEQ, micrófonos, construcción de cables XLR a medida.  
}
\cvcomputer{\bfseries{Micro Controladores}}{
	Programación de microcontroladores PIC16F84, PIC16F877, ARM.
}{\bfseries{Microondas}}{
	Fundamentos básicos de microondas, transmisión y propagación de ondas.
}
\cvcomputer{\bfseries{Antenas}}{
	Construcción de antenas del tipo: Yagi, Omni, Guia de ondas sectorial y Biquad para las frecuencias 2.4 y 5.8 Ghz.
}{\bfseries{Sistemas Satelitales}}{
	Instalación, operación y mantenimiento de las estaciones remotas del \bfseries{Satélite Simón Bolivar}
}

\section{Experiencia}

\cventry{Actualidad}{Contratista}{\bfseries{CANTV}}{Caracas}{ Migración a
Software Libre de los siguientes servicios: Plataforma de Correo Corporativo
(Exchange), Plataforma DNS Corporativo, DHCP Corporativo, Autenticación y
Calendario Colaborativo}{Tecnologías Involucradas: 
{\scriptsize
\begin{itemize}
\item CFEngine3 
\item Postfix
\item Exchange 
\item Dovecot 
\item Debian GNU/Linux 6 
\item Microsoft Windows Server 2003 R2 
\item OpenVZ 
\item KVM 
\item PROXMOX
\item OpenLDAP 
\item Perl
\item LaTeX
\item Active Directory
\end{itemize}
}
}{}%

\cventry{Julio 2010--\\Actualidad}{Contratista}{\bfseries{Centro Nacional de Tecnologías de Información (CNTI)}}{San Cristóbal}{Soporte de 4to Nivel bajo la figura de acompañamiento para la Gerencia de Plataforma Tecnológica CNTI}{Tecnologías Involucradas: XEN, KVM, PROXMOX, OCS Inventory, PostgreSQL, Mysql, Perl, PHP, Asesoría para la creación de planes de prevención y contingencia de los servicios, Mantener la disponibilidad de los servicios de las plataformas, Correo, DHCP, DNS, Firewalls, OpenLDAP, Puppet, CFEngine, Trac, SVN, GIT, MoinMoin, Request Tracker, OpenLDAP + Samba, VPN, Firewall, Infraestructura de PKI, Aseguramiento de aplicaciones, Infraestructura de Autenticación y Autorización LDAP, Pruebas de Penetración, Assesment de Vulnerabilidades, Hardening de Servicios, SGSI + ISO 27001-2, Investigación Forense, Telefonía IP, Nagios, Zabbix, OSSIM, Heartbeat V2, DRBD, Linux-HA, OCFS2, iSCSI, Cluster XEN, Open Cluster Framework.}{}

\cventry{Diciembre 2007--\\Actualidad}{Coordinador General}{\bfseries{Asociación Cooperativa Venezolana de Tecnologías Libres R.S. (COVETEL)}}{San Cristóbal}{ Responsable de las siguientes áreas: Coordinación, Gerencia, Ventas, Proyectos }{}{}

\cventry{Julio 2010--\\Actualidad}{Contratista}{\bfseries{Centro Nacional de Tecnologías de Información (CNTI)}}{San Cristóbal}{Soporte de 4to Nivel bajo la figura de acompañamiento para la Gerencia de Plataforma Tecnológica CNTI}{Tecnologías Involucradas: XEN, KVM, PROXMOX, OCS Inventory, PostgreSQL, Mysql, Perl, PHP, Asesoría para la creación de planes de prevención y contingencia de los servicios, Mantener la disponibilidad de los servicios de las plataformas, Correo, DHCP, DNS, Firewalls, OpenLDAP, Puppet, CFEngine, Trac, SVN, GIT, MoinMoin, Request Tracker, OpenLDAP + Samba, VPN, Firewall, Infraestructura de PKI, Aseguramiento de aplicaciones, Infraestructura de Autenticación y Autorización LDAP, Pruebas de Penetración, Assesment de Vulnerabilidades, Hardening de Servicios, SGSI + ISO 27001-2, Investigación Forense, Telefonía IP, Nagios, Zabbix, OSSIM, Heartbeat V2, DRBD, Linux-HA, OCFS2, iSCSI, Cluster XEN, Open Cluster Framework.}{}

\cventry{Junio 2010--\\Actualidad}{Contratista}{\bfseries{Centro Nacional de Tecnologías de Información (CNTI)}}{San Cristóbal}{Desarrollo del Sistema para la Elaboración Automatizada de Carnets Institucionales basado en LDAP}{Tecnologías Involucradas: OpenLDAP, Perl, Catalyst Framework, HTML::FormFu, Moose, REST, GD, File::Tar, XHTML, CSS, AJAX}{}

\cventry{Enero 2010--\\Actualidad}{Contratista}{\bfseries{Centro Nacional de Tecnologías de Información (CNTI)}}{San Cristóbal}{Desarrollo del Sistema para la Automatización del protocolo de prueba de portales de Internet | Nota: Este proyecto lo desarrolla el equipo de Covetel al cual pertenezco}{Tecnologías Involucradas: Perl, Catalyst Framework, HTML::FormFu, Moose, PostgreSQL, DBIx::Class, REST, SOAP, WSDL, OpenLDAP}{}

\cventry{Enero 2010--\\Actualidad}{Contratista}{\bfseries{Compañía Anónima Nacional Teléfonos de Venezuela (CANTV)}}{San Cristóbal}{Despliegue Satelital Simón Bolívar}{Responsabilidades: Instalación, Puesta a Punto y Mantenimiento de las terminales terrestres del Satélite Simón Bolivar en el estado Táchira}{}

\cventry{Noviembre 2009--\\Diciembre 2009}{Contratista}{\bfseries{Compañía Anónima Nacional Teléfonos de Venezuela (CANTV)}}{Caracas}{Fue Dictado un Curso Especializado e Intensivo de Xen - LDAP - Correo - Samba - Alta Disponibilidad al Equipo del Proyecto de Migración a Software Libre CANTV}{Tecnologías Involucradas: XEN, OpenLDAP, Postfix, Dovecot, Samba}{}

\cventry{Mayo 2009--\\Agosto 2009}{Contratista}{\bfseries{Compañía Anónima Nacional Teléfonos de Venezuela (CANTV) | Presidencia de la República Bolivariana de Venezuela}}{Caracas}{Asesorías para la creación de una plataforma de Telefonía IP basada en Asterisk con SER, escalable para la Presidencia de la República Bolivariana de Venezuela}{Tecnologías Involucradas: XEN, OpenLDAP, Postfix, Dovecot, Samba}{}

\cventry{Noviembre 2008--\\Enero 2010}{Contratista}{\bfseries{Ministerio del Poder Popular para Economía y Finanzas}}{Caracas}{Migración y Acondicionamiento de la Plataforma de Servicios}{Actividades (Realizadas por el equipo de Covetel, al cual pertenezco): Implementación de un Laboratorio de Migración, Adecuación de la Plataforma de Virtualización, Migración de cuentas de usuario desde AD Windows 2000 a OpenLDAP, Migración de los buzones al nuevo servidor de correo, Implementación de un Proxy con control parental, Implementación de un servidor de correo Postfix + Dovecot, Implementación servidor DNS Primario utilizando Bind integrado a OpenLDAP, Implementación de un servidor DNS Cache, Implementación de un servidor SAMBA PDC, Transferencia Tecnológica y Soporte a la plataforma durante ese periodo
 }{}

\cventry{Junio 2009--\\Diciembre 2009}{Contratista}{\bfseries{Centro Nacional de Tecnologías de Información (CNTI)}}{Caracas}{Construcción del Kit de Servicios, Nota: Este proyecto fue realizado por el equipo de Covetel, al cual pertenezco}{Partes del Kit de Servicios: Xen, OpenLDAP, TinyCA, Postfix, Dovecot, Thunderbird, RoundCube, Spamassassin, Dspam, ClamAV, Amavis-new, NTP, Firewall (Iptables), OpenVPN, DNS Integrado a LDAP, DHCP Integrado a LDAP, FreeRadius y autenticación 802.1X, Apache2 (Autenticación y VHOST integrados a LDAP), Samba (PDC, BDC, Servidor de Impresion y Servidor de Archivos), FAI, Puppet, SystemImager}{}

\cventry{Julio 2009--\\Agosto 2009}{Apoyo Voluntario}{\bfseries{Superintendencia de Servicios de Certificación Electrónica}}{Caracas}{Asesoría de seguridad durante la implementación de los servicios web del proyecto para consejos comunales}{Tecnologías Involucradas: Firewall, MySQL, PHP5, Hardening, Balanceo de Carga con DNS usando multiples punteros de tipo A }{}

\cventry{Noviembre 2006--\\Diciembre 2007}{Asesor Externo}{\bfseries{Distrito Tecnológico Social AIT PDVSA Mérida}}{Mérida}{Implementación de la Infraestructura de Networking y asesor en las siguientes áreas: Seguridad Lógica, Servicios de Red, Redes Inalámbricas}{}{}

\cventry{Noviembre 2004--\\Julio 2006}{Desarrollador}{\bfseries{IntranetHome C.A. - Contratista DirectTV}}{San Cristóbal}{Desarrollador Líder, de la aplicación de control de instalaciones y servicios para las contratistas de DIRECTV de la región Occidental}{Tecnologías Involucradas: PHP5, JAVA SCRIPT, PERL, XHTML, CSS, SQL.}{}

\cventry{Durante el año 2006}{Facilitator}{\bfseries{Universidad Valle del Momboy}}{Valera - Trujillo}{Diplomado de Software Libre}{}{}

\cventry{Noviembre 2005--Diciembre 2005}{Administrador de Red}{\bfseries{Juegos Nacionales Andes 2005}}{San Cristóbal}{Implementación y administración de los servidores que controlaban la conectividad de las infraestructuras deportivas}{}{}

\cventry{Durante el año 2004}{Miembro Fundador}{\bfseries{Cooperativa de Investigación y Desarrollo de Telecomunicaciones CIDTEL 546}}{San Cristóbal}{}{}{}

\cventry{2000--2002}{Administrador de Red}{AirNet C.A.}{San Cristóbal}{Responsabilidades: Administrar una red inalámbrica implementada usando el estandar 802.11b, utilizando Linux Router Proyect como terminales remotos para los clientes}{Tecnologías Involucradas: Ipchains, Iptables, LRP, Kernel Linux 2.2.9, Squid, DNS}{}


\section{Eventos}

\cventry{17 de Noviembre del 2010}{Ponente}{Evento de Telefonía IP}{San Cristóbal / Instiruto Universitario de Tecnologia Agro-Industrial}{Tema: Telefonía IP con Software Libre - Asterisk (Dumar Ramírez y Walter Vargas)}{}{}

\cventry{20 y 21 de Julio del 2007}{Ponente}{3er Congreso Nacional de Software Libre (CNSL 3)}{Zulia / Museo de Arte del Zulia}{Tema: Redes Inalámbricas con GNU/Linux}{}{}

\cventry{15 y 16 de Junio del 2007}{Ponente}{3er Congreso Nacional de Software Libre (CNSL 3)}{Mérida / ULA}{Tema: Redes Inalámbricas con GNU/Linux}{}{}
\cventry{08 y 09 de Junio del 2007}{Ponente}{3er Congreso Nacional de Software Libre (CNSL 3)}{Barinas / UNELLEZ}{Tema: Redes Inalámbricas con GNU/Linux}{}{}\cventry{08 y 09 de Junio del 2007}{Ponente}{3er Congreso Nacional de Software Libre (CNSL 3)}{Barinas / UNELLEZ}{Tema: Redes Inalámbricas con GNU/Linux}{}{}
\cventry{26 al 28 de Mayo del 2006}{Ponente}{2do Congreso Nacional de Software Libre (CNSL 2)}{Barquisimeto / Lara}{Tema: Redes Inalámbricas con GNU/Linux}{}{}
\cventry{19 al 21 de Mayo del 2006}{Ponente}{2do Congreso Nacional de Software Libre (CNSL 2)}{Valera / Trujillo}{Tema: Redes Inalámbricas con GNU/Linux}{}{}
\cventry{12 al 14 de Mayo del 2006}{Ponente}{2do Congreso Nacional de Software Libre (CNSL 2)}{Mérida}{Tema: Redes Inalámbricas con GNU/Linux}{}{}
\cventry{5 al 7 de Mayo del 2006}{Ponente}{2do Congreso Nacional de Software Libre (CNSL 2)}{Valera / Trujillo}{Tema: Redes Inalámbricas con GNU/Linux}{}{}
\cventry{28 al 30 de Abril del 2006}{Ponente}{2do Congreso Nacional de Software Libre (CNSL 2)}{San Cristóbal / Táchira}{Tema: Redes Inalámbricas con GNU/Linux}{}{}
\cventry{18 al 20 de Enero}{Ponente}{VI Congreso de Expotecnología UVM}{Valera / Trujillo}{Tema: Redes Inalámbricas con GNU/Linux}{}{}
\cventry{30 de Junio y 1 de Julio 2005}{Ponente}{1er Congreso Nacional de Software Libre}{Mérida}{Tema: Redes Inalámbricas con GNU/Linux}{}{}
\cventry{20 y 21 de Junio 2005}{Ponente}{1er Congreso Nacional de Software Libre}{Barinas}{Tema: Redes Inalámbricas con GNU/Linux}{}{}
\cventry{15 de Marzo 2005}{Ponente}{III Congreso de Gerencia Informática}{San Cristóbal}{Tema: Redes Inalámbricas con GNU/Linux}{}{}
\cventry{26 de Febrero 2005}{Ponente}{2do Foro Regional de Tecnología Libre}{San Cristóbal}{Tema: Redes Inalámbricas con GNU/Linux}{}{}
\cventry{26 de Febrero 2005}{Organizador}{2do Foro Regional de Tecnología Libre}{San Cristóbal}{Tema: Redes Inalámbricas con GNU/Linux}{}{}
\cventry{12 de Noviembre 2004}{Organizador}{1er Foro Regional de Tecnología Libre}{San Cristóbal}{Tema: Redes Inalámbricas con GNU/Linux}{}{}

\section{Intereses}
\cvitem{Diseño}{\small Amante del Diseño y la Fotografía.}
\cvitem{Deportes}{\small Natación, Montañismo, Parapente, Paracaidismo, Motocross}
\cvitem{Bricolaje}{\small Amante de las maquinas y herramientas para trabajo con madera, metal y plastico}

% Publications from a BibTeX file without multibib\renewcommand*{\bibliographyitemlabel}{\@biblabel{\arabic{enumiv}}}% for BibTeX numerical labels
\nocite{*}
\bibliographystyle{plain}
\bibliography{publications}       % 'publications' is the name of a BibTeX file

% Publications from a BibTeX file using the multibib package
%\section{Publications}
%\nocitebook{book1,book2}
%\bibliographystylebook{plain}
%\bibliographybook{publications}   % 'publications' is the name of a BibTeX file
%\nocitemisc{misc1,misc2,misc3}
%\bibliographystylemisc{plain}
%\bibliographymisc{publications}   % 'publications' is the name of a BibTeX file

\end{document}


%% end of file `walter.tex'.
